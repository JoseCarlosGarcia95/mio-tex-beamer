\begin{frame}
	\centering \LARGE \color{naranjaUCA} Modelo de cola M/G/1
\end{frame}
\begin{frame}{M/G/1}
Debido a que $G$ no es exponencial no podemos modelar este tipo de colas un proceso de nacimiento-muerte. \\ \pause
Utilizamos los resultados de Pollaczek y Khinchin para determinar $L_q$, $L$, $L_s$, $W_q$, $W$, $W_s$ \pause


\begin{equation}
L_q=\frac{\lambda^2 \sigma^2+\rho}{2(1-\rho)} ~ ~ ~ \rho=\frac{\lambda}{\mu}
\end{equation}
Como $W_s=\displaystyle\frac{1}{\mu}$, por Little tenemos:
$L_s=\lambda\left(\displaystyle\frac{1}{\mu}\right)=\rho$. Como $L=L_s+L_q$ obtenemos que
\begin{equation}
L=L_q+\rho
\end{equation}


\end{frame}

\begin{frame}{M/G/1}
Nuevamente de las fórmulas de Little tenemos

$$
W_q=\frac{L_q}{\lambda}$$
$$
W=W_q+\frac{1}{\mu}
$$

\end{frame}
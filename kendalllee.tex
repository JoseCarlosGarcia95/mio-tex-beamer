\begin{frame}
	\centering \LARGE \color{naranjaUCA} Kendall-Lee
\end{frame}
\begin{frame}{Kendall-Lee}
Cada sistema de líneas de espera se describe mediante seis características:
1/2/3/4/5/6
\\
		La primera especifica la naturaleza del proceso de llegada. Se utilizan las abreviaturas siguientes:
		\pause
		\begin{itemize}
			\item M=Los tiempos entre llegadas son variables aleatorias independientes e idénticamente distribuidas (iid) cuya distribución es exponencial.
			\pause
			\item D= Los tiempos entre llegadas son iid y deterministas. \pause
			\item $E_k$=Los tiempos entre llegadas son Erlangs iid con parámetro de forma k.
			\pause
			\item GI=Los tiempos entre llegadas son iid y están regidos por alguna distribución general.
			\pause
		\end{itemize}

\end{frame}

\begin{frame}{Kendall-Lee}
	
	La segunda característica especifica la naturaleza de los tiempos de servicio.
	\pause
	\\
	La tercera característica es la cantidad de servidores en paralelo. 
	\pause
	\\ La cuarta característica es la disciplina de líneas de espera.
	\pause
	\\	La quinta denota la capacidad del sistema.
	\\
	\pause
	
	La sexta es el tamaño de la población de donde se extraen los clientes.
\end{frame}
